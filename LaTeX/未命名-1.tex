\documentclass[a4paper,11pt]{article}
\usepackage{amsfonts}
\usepackage{amsmath}
\usepackage{makeidx}
\makeindex
\newtheorem{law}{Law}
\newtheorem{mur}{Murphy}[subsection]
\newtheorem{jury}[law]{Jury}
\pagestyle{headings}
\author{Rui Sun}
\title{Minimalism}
\begin{document}
\maketitle
\tableofcontents
\section{Start}
I read that Knuth divides the
people working with \TeX{} into
\TeX{}nicians and \TeX perts.\\
Today is \today

%Example1
\ldots when Einstein introduced his formula
\begin{equation}
    e = m \cdot c^2 \; ,
\end{equation}
which is at the same time the most widely known 
and the least well understood physical formula.

%\section{Example2}
\ldots from which follows Kirchoff's current law
\begin{equation}
    \sum_{k=1}^{n} I_k = 0\; .
\end{equation}
Kirchoff's voltage law can be derived \ldots

%\section{Example3}
\ldots which has several advantages.

\begin{equation}
    I_D = I_F - I_R
\end{equation}
is the core of a very different transistor model. \ldots

shelfful\\
shelf\mbox{}ful \\
Mr.~Smith was happy to see her\\
cf.~Fig.~5\\
I like BASIC\@. What about you?\\

Footnotes\footnote{This 
is a footnote.} are often understood
by people using \LaTeX.

\emph{If you use emphasizing 
inside a piece of emphasized text, 
then \LaTeX{} uses the \emph{normal} font for emphasizing.}

\section{Document Layout}
\subsection{Emphasized}
\textit{You can also 
\emph{emphasized} text if 
it is set in italic,}
\textsf{in a 
\emph{sans-serif} font,}
\texttt{or in
\emph{typewriter} style.}

\subsection{Environment}
\subsubsection{Itemize, Enumerate, and Description}
\flushleft
\begin{enumerate}
    \item You can mix the list
environment to your taste:
    \begin{itemize}
        \item But it might start to
    look silly.
        \item[-] With a dash. 
    \end{itemize} 
    \item Therefore remember: 
    \begin{description}
        \item[Stupid] things will not become smart because they are in a list 
        \item[Smart] things, though, can be presented beautifully in a list.
    \end{description}
\end{enumerate}
\subsubsection{Flushleft, Flushright, and Center}
\begin{flushleft}
    This text is\\ left-aligned.
    \LaTeX{} is not trying to make each line
    the same length.
\end{flushleft}

\begin{flushright}
    This text is right-\\aligned.
    \LaTeX{} is not trying to make each
    line the same length.
\end{flushright}

\begin{center}
    At the center\\of the earth.
\end{center}

This text is default.\\
\LaTeX{} is not trying to make each line
the same length.

\subsubsection{Quote, Quotation, and Verse}
A typographical rule of thumb
for the line length is:
\begin{quote}
    On average, no line should
    be longer than 66 characters.
\end{quote}
This is why \LaTeX{} pages have such large borders
by default and also why multicolumn print is used in
newspaper.


I know only one English poem by heart.
It is about Humpty Dumpty.
\begin{flushleft}
    \begin{verse}
        Humpty Dumpty sat on a wall:\\
        Humpty Dumpty had a great fall.\\
        All the King's horses and all the
        King's men\\
        Couldn't put Humpty together again.
    \end{verse}
\end{flushleft}
\subsubsection{Print words by words}
The \verb|\ldots| command \ldots
\begin{verbatim}
    10 PRINT "HELLO WORLD" ;
    20 GOTO 10
\end{verbatim}

\begin{verbatim*}
    the starred version of 
    the     verbatim
    environment emphasizes
    the spaces in the text
\end{verbatim*}
\subsubsection{Table}
\begin{tabular}{|r|l|}
    \hline
    7C0 & hexadecimal \\
    3700 & octal \\ \cline{2-2}
    11111000000 & binary \\
    \hline \hline1984 & decimal \\
    \hline
\end{tabular}

\begin{tabular}{|p{4.7cm}|}
    \hline
    Welcome to Boxy's paragraph.
    We sincerely hope you'll
    all enjoy the show.\\
    \hline
\end{tabular}
\begin{tabular}{@{} l @{}}
    \hline
    no leading space\\
    \hline
\end{tabular}
\begin{tabular}{l}
    \hline
    leading space left and right\\
    \hline
\end{tabular}
\begin{tabular}{c r @{.} l}
    Pi expression &
    \multicolumn{2}{c}{Value} \\
    \hline
    $\pi$               & 3&1416    \\
    $\pi^{\pi}$         & 36&46     \\
    $\pi^{\pi^{\pi}}$   & 80662&7   \\
\end{tabular}
\begin{tabular}{|c|c|}
    \hline
    \multicolumn{2}{|c|}{Ene} \\
    \hline
    Mene & Muh! \\
    \hline
\end{tabular}
\subsection{Protect Command}
\subsection{I am considerate
\protect\footnote{and protect my footnotes}}

\section{Mathematics}
\subsection{Basic Knowledge}
Add $a$ squared and $b$ squared to get
$c$ squared. Or, using a more mathematical
approach: $c^2=a^2+b^2$ \\
\TeX{} is pronounced as $\tau\epsilon\chi$. \\[6pt]
100~m$^{3}$ of water\\[6pt]
This comes from my $\heartsuit$\\
Add $a$ squared and $b$ squared to get $c$ squared.
Or, using a more mathematical approach:
\begin{displaymath}
    c^2=a^2+b^2
\end{displaymath}
And just one more line.\\
\begin{equation}
    \label{eq:eps}
    \epsilon > 0    
\end{equation}
From (\ref{eq:eps}), we gather \ldots\\
$\lim_{n \to \infty}
\sum_{k=1}^n \frac{1}{k^2}
= \frac{\pi^2}{6}
$
\begin{displaymath}
    \lim_{n \to \infty}
    \sum_{k=1}^n\frac{1}{k^2}
    = \frac{\pi^2}{6}
\end{displaymath}
\begin{equation}
    \forall x \in \mathbf{R}:
    \qquad x^2 \geq 0
\end{equation}
\begin{equation}
    x^2 \geq 0\qquad
    \textrm{for all }x \in \mathbf{R}
\end{equation}
\begin{displaymath}
    x^2 \geq 0 \qquad
    \textrm{for all } x \in \mathbb{R}
\end{displaymath}
\subsection{Groups}
\begin{equation}
    a^x+y \neq a^{x+y}
\end{equation}
\subsection{Set mathematical formula module}
$\lambda, \xi, \pi, \mu, \Phi, \Omega, \alpha$ \\
$a_{1}$ \qquad $x^{2}$ \qquad
$e^{-\alpha t}$ \qquad
$a^{3}_{ij}$ \\
$e^{x^2} \neq ({e^x})^2 $
$\sqrt{x}$ \qquad
$\sqrt{x^{2}+\sqrt{y}}$
\qquad $\sqrt[3]{2}$\\[3pt]
$\surd[x^2 + y^2]$\\
$\overline{m+n}$ \qquad
$\underline{m+n}$\\
$\underbrace{a+b+\cdots+z}_{26}$\\
\begin{displaymath}
    y=x^2 \qquad y'=2x\qquad y''=2
\end{displaymath}
\begin{displaymath}
    \vec a\quad\overrightarrow{AB}
\end{displaymath}
\begin{displaymath}
    v={\sigma}_1 \cdot {\sigma}_2{\tau}_1\cdot{\tau}_2
\end{displaymath}
\[\lim_{x\to0}\frac{\sin x}{x}=1
    \]
\begin{displaymath}
    {n \choose k} \qquad {x \atop y+2} \qquad
    \int f_N(x) \stackrel{!}{=} 1 \qquad 
    {a,b,c} \neq \{a,b,c\}
\end{displaymath} 
\begin{displaymath}
    1 + \left( \frac{1}{1-x^2}
    \right) ^ 3
\end{displaymath}
\begin{displaymath}
    \Big((x+1)(x-1)\Big) ^ 2 \qquad
    \big(\Big(\bigg(\Bigg(\Bigg)\bigg)\Big)\big)
\end{displaymath}
\subsection{Math White Space}
\newcommand{\ud}{\mathrm{d}}
\begin{displaymath}
    \int\!\!\!\int_{D} g(x,y)
    \, \ud x\, \ud y
\end{displaymath}
instead of
\begin{displaymath}
    \int\int_{D} g(x,y)\ud x\ud y \qquad
    \iint 
\end{displaymath}
\subsection{Vertically Aligned}
\begin{displaymath}
    \mathbf{X} =
    \left(
        \begin{array}{ccc}
            x_{11} & x_{12} & \ldots \\
            x_{21} & x_{22} & \ldots \\
            \vdots & \vdots & \ddots
        \end{array}
    \right)
\end{displaymath}
\begin{displaymath}
    y = \left\{
        \begin{array}{ll}
            a   &   \textrm{if $d > c$}\\
            b+x &   \textrm{in the morning} \\
            l   &   \textrm{all day long}
        \end{array}
        \right.
\end{displaymath}
\begin{displaymath}
    \left(
        \begin{array}{c|c}
            1 & 2 \\
            \hline
            3 & 4
        \end{array}
    \right)
\end{displaymath}
\begin{eqnarray}
    f(x)    &=& \cos x \\
    f'(x)   &=& -\sin x\\
    \int_{0}^{x} f(y)\ud y &=& \sin x
\end{eqnarray}
{\setlength\arraycolsep{2pt}
\begin{eqnarray}
    \sin x = x - \frac{x^3}{3!} + \frac{x^5}{5!}-{}
    \nonumber \\
    & & {}-\frac{x^7}{7!}+{}\cdots
\end{eqnarray}
}
\begin{eqnarray}
    \lefteqn{
        \cos x =1-\frac{x^2}{2!}+{}}
        \nonumber\\
        & &{}+\frac{x^4}{4!}-\frac{x^6}{6!}+{}\cdots
\end{eqnarray}

\subsection{Phatom}
\begin{displaymath}
    {}^{12}_{\phantom{1}6}\textrm{C}
    \qquad \textrm{versus}  \qquad
    {}^{12}_{6}\textrm{C}
\end{displaymath}
\begin{displaymath}
    \Gamma_{ij}^{\phantom{ij}k}
    \qquad \textrm{versus} \qquad
    \Gamma_{ij}^k
\end{displaymath}

\subsection{Math Font} 
\begin{equation}
    2^{\textrm{nd}} \quad
    2^{\mathrm{nd}} 
\end{equation}

\begin{displaymath}
    \mathop{\mathrm{corr}}(X,Y)=
    \frac{\displaystyle
        \sum^{n}_{i=1}(x_i-\overline{x})(y_i-\overline{y})}
    {\displaystyle\biggl[\sum_{i=1}^n (x_i - \overline{x})^2
    \sum_{i=1}^n (y_i - \overline{y})^2
    \biggr]^{1/2}}
\end{displaymath}

\subsection{Definition}
\begin{law} \label{law:box}
    Dont' hide in the witness box
\end{law}

\begin{jury}[The Twelve]
It could by you! So beware and
see law~\ref{law:box}
\end{jury}
\begin{law}No, No, No\end{law}

\begin{flushleft}
    \begin{mur}
        If there are two or more
        ways to do something, and one
        of those ways can result in a catastrophe
        , then someone will do it.
    \end{mur}
\end{flushleft}

\subsection{Bold Symbols}

\begin{displaymath}
    \mu, M \qquad \mathbf{M} \qquad
    \mbox{\boldmath $\mu, M$}
\end{displaymath}

\begin{displaymath}
    \mu, M \qquad
    \boldsymbol{\mu}, \boldsymbol{M}
\end{displaymath}

\section{Special Function}
\subsection{Bibliography}
Part1~\cite{pa} has
proposed that \ldots
\begin{thebibliography}{99}
    \bibitem{pa} H.~Part1:
    \emph{German\TeX}
    TUGboat Volume~9, Issue~1 (1998)
\end{thebibliography}
\subsection{Index}
\index{hello}
\section{DIY LaTeX}
\subsection{Set up new command/environment/package}
\subsubsection{Set up new command}
\newcommand{\tnss}{The not
so Short Introduction to \LaTeX}
This is ''\tnss'' \ldots{} ''\tnss''
\newcommand{\txsit}[1]
{
    This is the \emph{#1} Short
    Introduction to \LaTeX}
\begin{itemize}
    \item \txsit{not so}
    \item \txsit{very}
\end{itemize}

\subsubsection{Set up new environment}
\newenvironment{king}
{
    \rule{1ex}{1ex}%
    \hspace{\stretch{1}}
}
{
    \hspace{\stretch{1}}%
    \rule{1ex}{1ex}
}
\begin{king}
    My humble subjects \ldots
\end{king}
\subsubsection{Set up new package}
\subsection{Font and Size}
\subsubsection{Font Transform commands}
{
    \small The small and
    \textbf{bold} Romans ruled
}
{
    \Large all of great big
    \textit{Italy}.
}\\

He likes
{
  \Large large and \small small  
} letters.
\begin{Large}
    This is not true.
    But then again, what is these
    days \ldots
\end{Large}
\subsection{Seperators Between Objects}
\subsubsection{Row Distance}
\subsubsection{Paragraph}
\subsubsection{Surface Distance}
This is \hspace{1.5cm} a white space with 1.5cm.\\
x\hspace{\stretch{1}}
x\hspace{\stretch{3}}x
\subsubsection{Vertical Distance}
\subsection{More Details of Length}
\begin{flushleft}
    \newenvironment{vardesc}[1]{
        \settowidth{\parindent}{#1: \ }
        \makebox[0pt][r]{#1: \ }}{}
    \begin{displaymath}
        a^2+b^2=c^2
    \end{displaymath}
    \begin{vardesc}{Where}$a$,
    $b$ -- are adjunct to the right angle of a 
    right-angled triangle.

    $c$ -- is the hypotenuse of the triangle
    and feels lonely.

    $d$ -- finally does not show up here
    at all. Isn't that puzzling?
    \end{vardesc}
\end{flushleft}
\subsection{Box}
\makebox[\textwidth]{
    c e n t r a l
}\par
\makebox[\textwidth][s]{
    s p r e a d
}\par
\framebox[1.1\width] {
    Guess I'm framed now!
}\par
\framebox[0.8\width][r]{Bummer, 
I am too wide} \par
\framebox[1cm][l]{never mind, so am I}
Can you read this?
\end{document}

