\documentclass{article}
\usepackage{amsmath}
\usepackage{amsfonts}
\usepackage{makeidx}
\usepackage{indentfirst}
\usepackage{listings}
\usepackage{fancyhdr}
\usepackage[UTF8]{ctex} % 显示中文

\pagestyle{fancy}
\renewcommand{\sectionmark}[1]{\markright{\thesection\ #1}}
\renewcommand{\paragraphmark}[1]{\markboth{#1}{}}
\fancyhf{}
\fancyhead[LE,RO]{\bfseries\thepage}
\fancyhead[LO]{\bfseries\rightmark}
\fancyhead[RE]{\bfseries\leftmark}
\addtolength{\headheight}{0.5pt}
\fancypagestyle{plain}{
    \fancyhead{}
    \renewcommand{\headrulewidth}{0pt}
}

\newtheorem{law}{Law}[section]

\makeindex
\author{Rui Sun}
\title{\LaTeX{} Tutorial}

\usepackage{xcolor}
\lstset{
  language=C++,  %代码语言使用的是matlab
  frame=shadowbox, %把代码用带有阴影的框圈起来
  rulesepcolor=\color{red!20!green!20!blue!20},%代码块边框为淡青色
  keywordstyle=\color{blue!90}\bfseries, %代码关键字的颜色为蓝色,粗体
  commentstyle=\color{red!10!green!70}\textit,    % 设置代码注释的颜色
  showstringspaces=false,%不显示代码字符串中间的空格标记
  numbers=left, % 显示行号
  numberstyle=\tiny,    % 行号字体
  stringstyle=\ttfamily, % 代码字符串的特殊格式
  breaklines=true, %对过长的代码自动换行
  extendedchars=false,  %解决代码跨页时,章节标题,页眉等汉字不显示的问题
%   escapebegin=\begin{CJK*},escapeend=\end{CJK*},      % 代码中出现中文必须加上,不然报错
  texcl=true}
%-----------------------------
\begin{document}
\maketitle
\tableofcontents
\section{Elementary Knowledge}
\subsection{Source Code}

\subsubsection{Space}
\begin{itemize}
    \item Continuous space characters
    are considered as one space character.
    \item The space characters at the beginning
    of each line will be ignored.
    \item Single LF(Line Feed) will be considered as
    a space.  
    \item \LaTeX{} uses empty line to end the paragraph.
    \item Multiple LF will be considered as one LF. 
\end{itemize}

\subsubsection{Special Characters}
These characters are special characters in \LaTeX{},
which are commonly used for special functions.
If want to display these characters, you should add $\backslash$ before characters. \\
\# \; \$ \; \% \; \^ \; \& \; \_ \; \{ \; \} \; 
\~ \; $\backslash$ \\
\emph{$\backslash\backslash$ is used to end the line.}
\footnote{\$$\backslash$backslash\$ will generate $\backslash$.}

\subsubsection{\LaTeX{} Commands}
\begin{itemize}
    \item Commands is case- sensitive.
    \item \LaTeX{} ignores the space behind the commands.
    If you want to add space behind the commands, you can 
    add \{\} and a space behind the commands. 
\end{itemize}

\subsubsection{Annotation}
\LaTeX{} will ignore all the context
behind \%

\section{Document Typesetting}
\subsection{Tokenize in words}
Command \framebox{$\backslash$mbox\{text\}} 
ensures to set multiple words on the same line.
\subsection{Special Characters}

\subsubsection{Dash and Hyphen}
\begin{itemize}
    \item -  : hyphen(one)
    \item -- :  short dash(two)
    \item --- : long dash(three)
\end{itemize}

\subsubsection{Tilde}
\begin{itemize}
    \item \~{} : \framebox{$\backslash$ \~{}}
    \item $\sim$ : \framebox{\$$\backslash$sim\$}
\end{itemize}

\subsubsection{Angle}
\begin{itemize}
    \item $^{\circ}\mathrm{C}$ : \framebox{
    \$\;\^\;\{$\backslash$ circ\}}
\end{itemize}

\subsubsection{Ellipsis}
\begin{itemize}
    \item \ldots{} : \framebox{
    $\backslash$ldots} 
\end{itemize}

\subsection{Words' Seperator}
\begin{itemize}
    \item The backslash before the space can
    generate a non-extended space 
    \item \~{} can generate a non-extended space
    and forbid newline
    \item \framebox{$\backslash$@} before period
    indicates the period is the end of sentence.
\end{itemize}

\subsection{Cross Quotation}
\framebox{$\backslash$label\{\emph{marker}\},
$\backslash$ref\{\emph{marker}\},
$\backslash$pageref\{\emph{marker}\}
}\\
A reference to this subsection \label{sec:this}
looks like: "see section~\ref{sec:this} on
page~\pageref{sec:this}."

\subsection{Footnote}
\framebox{$\backslash$footnote\{
    \emph{footnote text} \}
}

\subsection{Emphasize}

\begin{itemize}
    \item \underline{underline}{} : \framebox{$\backslash$underline\{\emph{text}\}}
    \item \emph{emphasize}{} : \framebox{$\backslash$emph\{\emph{text}\}}
\end{itemize}

\subsection{Environment}
\framebox{$\backslash$begin\{\emph{environment}\} 
    \; \emph{text} \; 
    $\backslash$end\{\emph{environment}\} 
}

\subsubsection{Itemize, Enumerate, and Description}
\begin{itemize}
    \item itemize : set up simple list
    \item enumerate : set up list with index
    \item description : set up list with description
\end{itemize}

\subsubsection{Flushleft, Flushright, and Center}
\begin{itemize}
    \item Flushleft : generate a paragraph arranged to the left
    \item Flushright :  generate a paragraph arranged to the right
    \item center : generate a paragraph arranged to the center.
\end{itemize}

\subsubsection{Quote, Quotation, and Verse}
\begin{itemize}
    \item quote : to quote phrase or examples
    \item quotation : to quote long paragraph
    \item verse : to quote poems 
\end{itemize}

\subsubsection{Table}
\framebox{$\backslash$begin\{tabular\}\{\emph{table spec}\}}
\emph{table spec}{} : \\
\begin{itemize}
    \item l : to generate a column arranged to the left
    \item r : to generate a column arranged to the right
    \item c : to generate a column arranged to the center
    \item | : to generate a plumb line
    \item p\{\emph{width}\}{} : to generate a column with specific width
\end{itemize}
In tabular environment, use \framebox{\&} to jump to next column; use
\framebox{$\backslash\backslash$} to start a new row; use \framebox{$\backslash$hline} to insert horizontal lines;
use \framebox{$\backslash$cline\{i-j\}} to insert part of horizontal line(i, j represent
the index of start column and end of column).

\section{Mathematics Formula}
\subsection{Elementary Knowledge}
It is supposed to set mathematics 
expression like :
\begin{itemize}
    \item \framebox{
        $\backslash$ (expression $\backslash$)}
    \item \framebox{\$ expression \$}
    \item \framebox{
        $\backslash$ begin\{math\}
        expression
        $\backslash$ end\{math\}}  
\end{itemize}
For large expression, it is suggested to use \emph{display} mode, like :
\begin{itemize}
    \item \framebox{
        $\backslash$ [
        expression
        $\backslash$ ]}
    \item \framebox{
        $\backslash$ begin\{displaymath\}
        expression
        $\backslash$ end\{diplaymath\}}  
\end{itemize}
However, this environment has no index. 
To append index for formula, you can use \emph{equation} environment. \\
There are differences between \emph{mathematics mode} between \emph{text mode}. In mathematics mode
\begin{enumerate}
    \item Space and FL will be ignored, space is instead by commands like
    \framebox{$\backslash$,}\;,\;\framebox{$\backslash$quad}
    or \framebox{$\backslash$qquad}.
    \item Empty line is forbidden. Each formula must belongs to only one paragraph.
    \item Every character will be considered as a variable name. 
    If you want to add normal text, you must use command 
    \framebox{$\backslash$textrm\{\emph{text}\}} to input text.
\end{enumerate}
Commonly, it is suggested to use blackboard bold to represent the set of 
real numbers. Use command \framebox{$\backslash$mathbb} to use this fonts.

\subsection{Commands}

\subsubsection{Greek letters}
Lowercase Greek letters : 
\framebox{$\backslash$alpha, 
$\backslash$beta,
$\backslash$gamma}{} , \ldots \\
\indent
Oppositely, to get Uppercase Greek letters :
Lowercase Greek letters : 
\framebox{$\backslash$Gamma, 
$\backslash$Delta,
$\backslash$Gamma}{} , \ldots 

\subsubsection{Square Root}
\begin{itemize}
    \item $\sqrt x$ : \framebox{$\backslash$sqrt x}
    \item $\sqrt[n] x$ : \framebox{
        $\backslash$sqrt[n] x
    }
    \item $\surd$ : \framebox{$\backslash$surd}
\end{itemize}

\subsubsection{Horizontal Line}
\begin{itemize}
    \item $\overline{m+n}$ : \framebox{
        $\backslash$overline\{m+n\}
    }
    \item $\underline{m+n}$ : \framebox{
        $\backslash$underline\{m+n\}
    }
    \item $\overbrace{a+b+\cdots+z}^{26}$ : \framebox{
        $\backslash$overbrace\{\ldots\}\;\^\;\{\ldots\}
    }
    \item $\underbrace{a+b+\cdots+z}_{26}$ : \framebox{
        $\backslash$underbrace\{\ldots\}\_\{\ldots\}
    }
\end{itemize}

\subsubsection{Arrow} 
\begin{itemize}
    \item $\vec a$ : \framebox{
        $\backslash$vec a
    }
    \item $\overleftarrow{AB}$ : \framebox{
        $\backslash$overleftarrow\{AB\}
    }
    \item $\overrightarrow{AB}$ : \framebox{
        $\backslash$overrightarrow\{AB\}
    }
\end{itemize}

\subsubsection{Binomial}
\begin{itemize}
    \item ${n \choose k}$ : \framebox{
        \{\ldots\;$\backslash$choose\;\ldots\}
    }
    \item ${n \atop k}$ : \framebox{
        \{\ldots\;$\backslash$atop\;\ldots\}
    }
\end{itemize}

\subsubsection{Binary Relationship}
\begin{itemize}
    \item $\stackrel{!}{=}$ : \framebox{
        $\backslash$stackrel\{!\}\{=\}
    }
\end{itemize}

\subsubsection{Brace Size} 
\begin{itemize}
    \item $\big(\Big(\bigg(\Bigg($ : \framebox{
        $\backslash$big(
        $\backslash$Big(
        $\backslash$bigg(
        $\backslash$Bigg(
    }
    \item $\big\}\Big\}\bigg\}\Bigg\}$ : \framebox{
        $\backslash$big\}
        $\backslash$Big\}
        $\backslash$bigg\}
        $\backslash$Bigg\}
    }
\end{itemize}

\subsubsection{Dots}
\begin{itemize}
    \item \ldots{} : \framebox{
        $\backslash$ldots
    }
    \item $\cdots$ : \framebox{
        $\backslash$cdots
    }
    \item $\vdots$ : \framebox{
        $\backslash$vdots
    }
    \item $\ddots$ : \framebox{
        $\backslash$ddots
    }
\end{itemize}

\subsection{Space} 
\begin{itemize}
    \item $\backslash$, : $\frac{3}{18}$quad\verb*| |
    \item $\backslash$: : $\frac{4}{18}$quad\verb*| |
    \item $\backslash$; : $\frac{5}{18}$quad\verb*| |
    \item $\backslash$\verb*| |: $\frac{1}{2}$quad\verb*| |
    \item $\backslash$quad : quad\verb*| |
    \item $\backslash$qquad : 2quad\verb*| |
\end{itemize}

\subsection{Vertical Alignment}
\subsubsection{Arrays}
\begin{itemize}
    \item $
        =\left( \begin{array}{ccc}
            x_{11} & x_{12} & \ldots \\
            x_{21} & x_{22} & \ldots \\
            \vdots & \vdots & \ddots
        \end{array}\right)
    $ 
    : \; \parbox{\textwidth}{
            =$\backslash$left( $\backslash$begin\{array\}\{ccc\} \par
            x\_\{11\} \& x\_\{12\} \& $\backslash$ldots $\backslash\backslash$ \par
            x\_\{21\} \& x\_\{22\} \& $\backslash$ldots $\backslash\backslash$ \par
            $\backslash$vdots \& $\backslash$vodts \& $\backslash$ddots \par
            $\backslash$end\{array\}$\backslash$right)
        }
        \\
        \\
    \item $
        =\left\{ \begin{array}{ll}
            a & \textrm{\ldots} \\
            b+x & \textrm{\ldots} \\
            l & \textrm{\ldots}
        \end{array}
        \right.
    $
    : \;\;\;\;\;\;\;\;\; \parbox{\textwidth}{
            =$\backslash$left$\backslash$\{ $\backslash$begin\{array\}\{ll\} \par
            a \& $\backslash$textrm\{$\backslash$ldots\} $\backslash\backslash$ \\
            b+x \& $\backslash$textrm\{$\backslash$ldots\} $\backslash\backslash$ \\
            l \& $\backslash$textrm\{$\backslash$ldots\} $\backslash\backslash$ \\
            $\backslash$end\{array\}$\backslash$right.
    }
    \\
    \\
    \item $
        \left(\begin{array}{c|c}
            1 & 2 \\
            \hline
            3 & 4
        \end{array}
        \right)
    $
    : \;\;\;\;\;\;\;\;\;\;\;\;\;\;\;\;\; \parbox{\textwidth}{
        =$\backslash$left( $\backslash$begin\{array\}\{c|c\} \par
        1 \& 2 $\backslash\backslash$ \\
        $\backslash$hline \\
        3 \& 4 $\backslash\backslash$ \\
        $\backslash$end\{array\}$\backslash$right)
    }
\end{itemize}

\subsubsection{Equations}
\begin{eqnarray}
    f(x) & = & \cos x \\
    f'(x) & = & -\sin x \\
    \int_{0}^x f(y)dy & = & \sin x
\end{eqnarray}
{\setlength\arraycolsep{2pt}
    \begin{eqnarray}
        \sin x & = & x -\frac{x^3}{3!} + \frac{x^5}{5!} - \nonumber \\
               &   & -\frac{x^7}{7!} + \cdots
    \end{eqnarray}
}
\begin{eqnarray}
    \lefteqn{
        \cos x = 1 - \frac{x^2}{2!} + {}} \nonumber \\
        & & {} + \frac{x^4}{4!} - \frac{x^6}{6!} + \cdots
\end{eqnarray}
\framebox{$\backslash$nonumber} will forbid \LaTeX{} to generate a index.

\subsection{Theory and Definitions}
\framebox{
    $\backslash$newtheorem\{name\}[counter]\{text\}[section]
}\\
Additionally, you should excute commands in preamble first, like :\\
\begin{law} 
    Don't hide in the witness box
\end{law}
\begin{lstlisting}
    \begin{law} 
        Don't hide in the witness box
    \end{law}
\end{lstlisting}

\subsection{bold}
It is hard to get bold characters in \LaTeX. You can
use command \framebox{$\backslash$mathbf}, but these ro$\mathbf{M}$an characters will be
vertical, however, mathematics symbols commanly are \emph{italic}. Here is one
command 
\framebox{
   $\backslash$boldmath  
} to get \emph{italics} fonts, but only valid in $\boldsymbol{M}$athematics mode.
Also the package \emph{amsbsy} and \emph{bm} can easily realize it, for they contain
the command \framebox{$\backslash$boldsymbol}

\section{DIY \LaTeX}
\subsection{Fonts and Size}
\begin{itemize}
    \item {\small text} : $\backslash$small
    \item {\Large text} : $\backslash$Large
    \item \textbf{bold face} : $\backslash$textbf
    \item \textit{italic} : $\backslash$textit
    \item \textrm{roman} : $\backslash$textrm
\end{itemize}

\subsection{Distance in Objects}
\subsubsection{Paragraph}
You can use \framebox{$\backslash$indent} to indent a unindent paragraph. \\
\indent
Also, you can use \framebox{$\backslash$noindent} to creat a non-indent paragraph.

\subsection{Box}
You can use \framebox{$\backslash$parbox[pos]\{\emph{width}\}\{\emph{text}\}}
to put a paragraph into a box. \\
\indent
Also, you can use \framebox{
    $\backslash$begin\{minipage\}[pos]
    \{\emph{width}\} text $\backslash$end
    \{minipage\} 
}
to do the same function.\\
\indent
command \framebox{
    $\backslash$makebox[\emph{width}][pos]
    \{\emph{text}\}
} has more powerful functions. \emph{Width} define the width observed from the
outside of the box. You can push $\backslash$width, $\backslash$height,
$\backslash$depth, to parameter. These variable values are obtained by measureing the context in the box.
\emph{Pos} accept 4 characters: \textbf{c}-center, \textbf{l}-left arragned, \textbf{r}-right arranged,
\textbf{s}-spread context evenly.
\end{document}